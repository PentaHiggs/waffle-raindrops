
% Default to the notebook output style

    


% Inherit from the specified cell style.




    
\documentclass{article}

    
    
    \usepackage{graphicx} % Used to insert images
    \usepackage{adjustbox} % Used to constrain images to a maximum size 
    \usepackage{color} % Allow colors to be defined
    \usepackage{enumerate} % Needed for markdown enumerations to work
    \usepackage{geometry} % Used to adjust the document margins
    \usepackage{amsmath} % Equations
    \usepackage{amssymb} % Equations
    \usepackage{eurosym} % defines \euro
    \usepackage[mathletters]{ucs} % Extended unicode (utf-8) support
    \usepackage[utf8x]{inputenc} % Allow utf-8 characters in the tex document
    \usepackage{fancyvrb} % verbatim replacement that allows latex
    \usepackage{grffile} % extends the file name processing of package graphics 
                         % to support a larger range 
    % The hyperref package gives us a pdf with properly built
    % internal navigation ('pdf bookmarks' for the table of contents,
    % internal cross-reference links, web links for URLs, etc.)
    \usepackage{hyperref}
    \usepackage{longtable} % longtable support required by pandoc >1.10
    \usepackage{booktabs}  % table support for pandoc > 1.12.2
    \usepackage{ulem} % ulem is needed to support strikethroughs (\sout)
    

    
    
    \definecolor{orange}{cmyk}{0,0.4,0.8,0.2}
    \definecolor{darkorange}{rgb}{.71,0.21,0.01}
    \definecolor{darkgreen}{rgb}{.12,.54,.11}
    \definecolor{myteal}{rgb}{.26, .44, .56}
    \definecolor{gray}{gray}{0.45}
    \definecolor{lightgray}{gray}{.95}
    \definecolor{mediumgray}{gray}{.8}
    \definecolor{inputbackground}{rgb}{.95, .95, .85}
    \definecolor{outputbackground}{rgb}{.95, .95, .95}
    \definecolor{traceback}{rgb}{1, .95, .95}
    % ansi colors
    \definecolor{red}{rgb}{.6,0,0}
    \definecolor{green}{rgb}{0,.65,0}
    \definecolor{brown}{rgb}{0.6,0.6,0}
    \definecolor{blue}{rgb}{0,.145,.698}
    \definecolor{purple}{rgb}{.698,.145,.698}
    \definecolor{cyan}{rgb}{0,.698,.698}
    \definecolor{lightgray}{gray}{0.5}
    
    % bright ansi colors
    \definecolor{darkgray}{gray}{0.25}
    \definecolor{lightred}{rgb}{1.0,0.39,0.28}
    \definecolor{lightgreen}{rgb}{0.48,0.99,0.0}
    \definecolor{lightblue}{rgb}{0.53,0.81,0.92}
    \definecolor{lightpurple}{rgb}{0.87,0.63,0.87}
    \definecolor{lightcyan}{rgb}{0.5,1.0,0.83}
    
    % commands and environments needed by pandoc snippets
    % extracted from the output of `pandoc -s`
    \providecommand{\tightlist}{%
      \setlength{\itemsep}{0pt}\setlength{\parskip}{0pt}}
    \DefineVerbatimEnvironment{Highlighting}{Verbatim}{commandchars=\\\{\}}
    % Add ',fontsize=\small' for more characters per line
    \newenvironment{Shaded}{}{}
    \newcommand{\KeywordTok}[1]{\textcolor[rgb]{0.00,0.44,0.13}{\textbf{{#1}}}}
    \newcommand{\DataTypeTok}[1]{\textcolor[rgb]{0.56,0.13,0.00}{{#1}}}
    \newcommand{\DecValTok}[1]{\textcolor[rgb]{0.25,0.63,0.44}{{#1}}}
    \newcommand{\BaseNTok}[1]{\textcolor[rgb]{0.25,0.63,0.44}{{#1}}}
    \newcommand{\FloatTok}[1]{\textcolor[rgb]{0.25,0.63,0.44}{{#1}}}
    \newcommand{\CharTok}[1]{\textcolor[rgb]{0.25,0.44,0.63}{{#1}}}
    \newcommand{\StringTok}[1]{\textcolor[rgb]{0.25,0.44,0.63}{{#1}}}
    \newcommand{\CommentTok}[1]{\textcolor[rgb]{0.38,0.63,0.69}{\textit{{#1}}}}
    \newcommand{\OtherTok}[1]{\textcolor[rgb]{0.00,0.44,0.13}{{#1}}}
    \newcommand{\AlertTok}[1]{\textcolor[rgb]{1.00,0.00,0.00}{\textbf{{#1}}}}
    \newcommand{\FunctionTok}[1]{\textcolor[rgb]{0.02,0.16,0.49}{{#1}}}
    \newcommand{\RegionMarkerTok}[1]{{#1}}
    \newcommand{\ErrorTok}[1]{\textcolor[rgb]{1.00,0.00,0.00}{\textbf{{#1}}}}
    \newcommand{\NormalTok}[1]{{#1}}
    
    % Additional commands for more recent versions of Pandoc
    \newcommand{\ConstantTok}[1]{\textcolor[rgb]{0.53,0.00,0.00}{{#1}}}
    \newcommand{\SpecialCharTok}[1]{\textcolor[rgb]{0.25,0.44,0.63}{{#1}}}
    \newcommand{\VerbatimStringTok}[1]{\textcolor[rgb]{0.25,0.44,0.63}{{#1}}}
    \newcommand{\SpecialStringTok}[1]{\textcolor[rgb]{0.73,0.40,0.53}{{#1}}}
    \newcommand{\ImportTok}[1]{{#1}}
    \newcommand{\DocumentationTok}[1]{\textcolor[rgb]{0.73,0.13,0.13}{\textit{{#1}}}}
    \newcommand{\AnnotationTok}[1]{\textcolor[rgb]{0.38,0.63,0.69}{\textbf{\textit{{#1}}}}}
    \newcommand{\CommentVarTok}[1]{\textcolor[rgb]{0.38,0.63,0.69}{\textbf{\textit{{#1}}}}}
    \newcommand{\VariableTok}[1]{\textcolor[rgb]{0.10,0.09,0.49}{{#1}}}
    \newcommand{\ControlFlowTok}[1]{\textcolor[rgb]{0.00,0.44,0.13}{\textbf{{#1}}}}
    \newcommand{\OperatorTok}[1]{\textcolor[rgb]{0.40,0.40,0.40}{{#1}}}
    \newcommand{\BuiltInTok}[1]{{#1}}
    \newcommand{\ExtensionTok}[1]{{#1}}
    \newcommand{\PreprocessorTok}[1]{\textcolor[rgb]{0.74,0.48,0.00}{{#1}}}
    \newcommand{\AttributeTok}[1]{\textcolor[rgb]{0.49,0.56,0.16}{{#1}}}
    \newcommand{\InformationTok}[1]{\textcolor[rgb]{0.38,0.63,0.69}{\textbf{\textit{{#1}}}}}
    \newcommand{\WarningTok}[1]{\textcolor[rgb]{0.38,0.63,0.69}{\textbf{\textit{{#1}}}}}
    
    
    % Define a nice break command that doesn't care if a line doesn't already
    % exist.
    \def\br{\hspace*{\fill} \\* }
    % Math Jax compatability definitions
    \def\gt{>}
    \def\lt{<}
    % Document parameters
    \title{customer\_segments}
    
    
    

    % Pygments definitions
    
\makeatletter
\def\PY@reset{\let\PY@it=\relax \let\PY@bf=\relax%
    \let\PY@ul=\relax \let\PY@tc=\relax%
    \let\PY@bc=\relax \let\PY@ff=\relax}
\def\PY@tok#1{\csname PY@tok@#1\endcsname}
\def\PY@toks#1+{\ifx\relax#1\empty\else%
    \PY@tok{#1}\expandafter\PY@toks\fi}
\def\PY@do#1{\PY@bc{\PY@tc{\PY@ul{%
    \PY@it{\PY@bf{\PY@ff{#1}}}}}}}
\def\PY#1#2{\PY@reset\PY@toks#1+\relax+\PY@do{#2}}

\expandafter\def\csname PY@tok@gd\endcsname{\def\PY@tc##1{\textcolor[rgb]{0.63,0.00,0.00}{##1}}}
\expandafter\def\csname PY@tok@gu\endcsname{\let\PY@bf=\textbf\def\PY@tc##1{\textcolor[rgb]{0.50,0.00,0.50}{##1}}}
\expandafter\def\csname PY@tok@gt\endcsname{\def\PY@tc##1{\textcolor[rgb]{0.00,0.27,0.87}{##1}}}
\expandafter\def\csname PY@tok@gs\endcsname{\let\PY@bf=\textbf}
\expandafter\def\csname PY@tok@gr\endcsname{\def\PY@tc##1{\textcolor[rgb]{1.00,0.00,0.00}{##1}}}
\expandafter\def\csname PY@tok@cm\endcsname{\let\PY@it=\textit\def\PY@tc##1{\textcolor[rgb]{0.25,0.50,0.50}{##1}}}
\expandafter\def\csname PY@tok@vg\endcsname{\def\PY@tc##1{\textcolor[rgb]{0.10,0.09,0.49}{##1}}}
\expandafter\def\csname PY@tok@m\endcsname{\def\PY@tc##1{\textcolor[rgb]{0.40,0.40,0.40}{##1}}}
\expandafter\def\csname PY@tok@mh\endcsname{\def\PY@tc##1{\textcolor[rgb]{0.40,0.40,0.40}{##1}}}
\expandafter\def\csname PY@tok@go\endcsname{\def\PY@tc##1{\textcolor[rgb]{0.53,0.53,0.53}{##1}}}
\expandafter\def\csname PY@tok@ge\endcsname{\let\PY@it=\textit}
\expandafter\def\csname PY@tok@vc\endcsname{\def\PY@tc##1{\textcolor[rgb]{0.10,0.09,0.49}{##1}}}
\expandafter\def\csname PY@tok@il\endcsname{\def\PY@tc##1{\textcolor[rgb]{0.40,0.40,0.40}{##1}}}
\expandafter\def\csname PY@tok@cs\endcsname{\let\PY@it=\textit\def\PY@tc##1{\textcolor[rgb]{0.25,0.50,0.50}{##1}}}
\expandafter\def\csname PY@tok@cp\endcsname{\def\PY@tc##1{\textcolor[rgb]{0.74,0.48,0.00}{##1}}}
\expandafter\def\csname PY@tok@gi\endcsname{\def\PY@tc##1{\textcolor[rgb]{0.00,0.63,0.00}{##1}}}
\expandafter\def\csname PY@tok@gh\endcsname{\let\PY@bf=\textbf\def\PY@tc##1{\textcolor[rgb]{0.00,0.00,0.50}{##1}}}
\expandafter\def\csname PY@tok@ni\endcsname{\let\PY@bf=\textbf\def\PY@tc##1{\textcolor[rgb]{0.60,0.60,0.60}{##1}}}
\expandafter\def\csname PY@tok@nl\endcsname{\def\PY@tc##1{\textcolor[rgb]{0.63,0.63,0.00}{##1}}}
\expandafter\def\csname PY@tok@nn\endcsname{\let\PY@bf=\textbf\def\PY@tc##1{\textcolor[rgb]{0.00,0.00,1.00}{##1}}}
\expandafter\def\csname PY@tok@no\endcsname{\def\PY@tc##1{\textcolor[rgb]{0.53,0.00,0.00}{##1}}}
\expandafter\def\csname PY@tok@na\endcsname{\def\PY@tc##1{\textcolor[rgb]{0.49,0.56,0.16}{##1}}}
\expandafter\def\csname PY@tok@nb\endcsname{\def\PY@tc##1{\textcolor[rgb]{0.00,0.50,0.00}{##1}}}
\expandafter\def\csname PY@tok@nc\endcsname{\let\PY@bf=\textbf\def\PY@tc##1{\textcolor[rgb]{0.00,0.00,1.00}{##1}}}
\expandafter\def\csname PY@tok@nd\endcsname{\def\PY@tc##1{\textcolor[rgb]{0.67,0.13,1.00}{##1}}}
\expandafter\def\csname PY@tok@ne\endcsname{\let\PY@bf=\textbf\def\PY@tc##1{\textcolor[rgb]{0.82,0.25,0.23}{##1}}}
\expandafter\def\csname PY@tok@nf\endcsname{\def\PY@tc##1{\textcolor[rgb]{0.00,0.00,1.00}{##1}}}
\expandafter\def\csname PY@tok@si\endcsname{\let\PY@bf=\textbf\def\PY@tc##1{\textcolor[rgb]{0.73,0.40,0.53}{##1}}}
\expandafter\def\csname PY@tok@s2\endcsname{\def\PY@tc##1{\textcolor[rgb]{0.73,0.13,0.13}{##1}}}
\expandafter\def\csname PY@tok@vi\endcsname{\def\PY@tc##1{\textcolor[rgb]{0.10,0.09,0.49}{##1}}}
\expandafter\def\csname PY@tok@nt\endcsname{\let\PY@bf=\textbf\def\PY@tc##1{\textcolor[rgb]{0.00,0.50,0.00}{##1}}}
\expandafter\def\csname PY@tok@nv\endcsname{\def\PY@tc##1{\textcolor[rgb]{0.10,0.09,0.49}{##1}}}
\expandafter\def\csname PY@tok@s1\endcsname{\def\PY@tc##1{\textcolor[rgb]{0.73,0.13,0.13}{##1}}}
\expandafter\def\csname PY@tok@kd\endcsname{\let\PY@bf=\textbf\def\PY@tc##1{\textcolor[rgb]{0.00,0.50,0.00}{##1}}}
\expandafter\def\csname PY@tok@sh\endcsname{\def\PY@tc##1{\textcolor[rgb]{0.73,0.13,0.13}{##1}}}
\expandafter\def\csname PY@tok@sc\endcsname{\def\PY@tc##1{\textcolor[rgb]{0.73,0.13,0.13}{##1}}}
\expandafter\def\csname PY@tok@sx\endcsname{\def\PY@tc##1{\textcolor[rgb]{0.00,0.50,0.00}{##1}}}
\expandafter\def\csname PY@tok@bp\endcsname{\def\PY@tc##1{\textcolor[rgb]{0.00,0.50,0.00}{##1}}}
\expandafter\def\csname PY@tok@c1\endcsname{\let\PY@it=\textit\def\PY@tc##1{\textcolor[rgb]{0.25,0.50,0.50}{##1}}}
\expandafter\def\csname PY@tok@kc\endcsname{\let\PY@bf=\textbf\def\PY@tc##1{\textcolor[rgb]{0.00,0.50,0.00}{##1}}}
\expandafter\def\csname PY@tok@c\endcsname{\let\PY@it=\textit\def\PY@tc##1{\textcolor[rgb]{0.25,0.50,0.50}{##1}}}
\expandafter\def\csname PY@tok@mf\endcsname{\def\PY@tc##1{\textcolor[rgb]{0.40,0.40,0.40}{##1}}}
\expandafter\def\csname PY@tok@err\endcsname{\def\PY@bc##1{\setlength{\fboxsep}{0pt}\fcolorbox[rgb]{1.00,0.00,0.00}{1,1,1}{\strut ##1}}}
\expandafter\def\csname PY@tok@mb\endcsname{\def\PY@tc##1{\textcolor[rgb]{0.40,0.40,0.40}{##1}}}
\expandafter\def\csname PY@tok@ss\endcsname{\def\PY@tc##1{\textcolor[rgb]{0.10,0.09,0.49}{##1}}}
\expandafter\def\csname PY@tok@sr\endcsname{\def\PY@tc##1{\textcolor[rgb]{0.73,0.40,0.53}{##1}}}
\expandafter\def\csname PY@tok@mo\endcsname{\def\PY@tc##1{\textcolor[rgb]{0.40,0.40,0.40}{##1}}}
\expandafter\def\csname PY@tok@kn\endcsname{\let\PY@bf=\textbf\def\PY@tc##1{\textcolor[rgb]{0.00,0.50,0.00}{##1}}}
\expandafter\def\csname PY@tok@mi\endcsname{\def\PY@tc##1{\textcolor[rgb]{0.40,0.40,0.40}{##1}}}
\expandafter\def\csname PY@tok@gp\endcsname{\let\PY@bf=\textbf\def\PY@tc##1{\textcolor[rgb]{0.00,0.00,0.50}{##1}}}
\expandafter\def\csname PY@tok@o\endcsname{\def\PY@tc##1{\textcolor[rgb]{0.40,0.40,0.40}{##1}}}
\expandafter\def\csname PY@tok@kr\endcsname{\let\PY@bf=\textbf\def\PY@tc##1{\textcolor[rgb]{0.00,0.50,0.00}{##1}}}
\expandafter\def\csname PY@tok@s\endcsname{\def\PY@tc##1{\textcolor[rgb]{0.73,0.13,0.13}{##1}}}
\expandafter\def\csname PY@tok@kp\endcsname{\def\PY@tc##1{\textcolor[rgb]{0.00,0.50,0.00}{##1}}}
\expandafter\def\csname PY@tok@w\endcsname{\def\PY@tc##1{\textcolor[rgb]{0.73,0.73,0.73}{##1}}}
\expandafter\def\csname PY@tok@kt\endcsname{\def\PY@tc##1{\textcolor[rgb]{0.69,0.00,0.25}{##1}}}
\expandafter\def\csname PY@tok@ow\endcsname{\let\PY@bf=\textbf\def\PY@tc##1{\textcolor[rgb]{0.67,0.13,1.00}{##1}}}
\expandafter\def\csname PY@tok@sb\endcsname{\def\PY@tc##1{\textcolor[rgb]{0.73,0.13,0.13}{##1}}}
\expandafter\def\csname PY@tok@k\endcsname{\let\PY@bf=\textbf\def\PY@tc##1{\textcolor[rgb]{0.00,0.50,0.00}{##1}}}
\expandafter\def\csname PY@tok@se\endcsname{\let\PY@bf=\textbf\def\PY@tc##1{\textcolor[rgb]{0.73,0.40,0.13}{##1}}}
\expandafter\def\csname PY@tok@sd\endcsname{\let\PY@it=\textit\def\PY@tc##1{\textcolor[rgb]{0.73,0.13,0.13}{##1}}}

\def\PYZbs{\char`\\}
\def\PYZus{\char`\_}
\def\PYZob{\char`\{}
\def\PYZcb{\char`\}}
\def\PYZca{\char`\^}
\def\PYZam{\char`\&}
\def\PYZlt{\char`\<}
\def\PYZgt{\char`\>}
\def\PYZsh{\char`\#}
\def\PYZpc{\char`\%}
\def\PYZdl{\char`\$}
\def\PYZhy{\char`\-}
\def\PYZsq{\char`\'}
\def\PYZdq{\char`\"}
\def\PYZti{\char`\~}
% for compatibility with earlier versions
\def\PYZat{@}
\def\PYZlb{[}
\def\PYZrb{]}
\makeatother


    % Exact colors from NB
    \definecolor{incolor}{rgb}{0.0, 0.0, 0.5}
    \definecolor{outcolor}{rgb}{0.545, 0.0, 0.0}



    
    % Prevent overflowing lines due to hard-to-break entities
    \sloppy 
    % Setup hyperref package
    \hypersetup{
      breaklinks=true,  % so long urls are correctly broken across lines
      colorlinks=true,
      urlcolor=blue,
      linkcolor=darkorange,
      citecolor=darkgreen,
      }
    % Slightly bigger margins than the latex defaults
    
    \geometry{verbose,tmargin=1in,bmargin=1in,lmargin=1in,rmargin=1in}
    
    

    \begin{document}
    
    
    \maketitle
    
    

    
    \section{Creating Customer Segments}\label{creating-customer-segments}

    In this project you, will analyze a dataset containing annual spending
amounts for internal structure, to understand the variation in the
different types of customers that a wholesale distributor interacts
with.

Instructions:

\begin{itemize}
\tightlist
\item
  Run each code block below by pressing \textbf{Shift+Enter}, making
  sure to implement any steps marked with a TODO.
\item
  Answer each question in the space provided by editing the blocks
  labeled ``Answer:''.
\item
  When you are done, submit the completed notebook (.ipynb) with all
  code blocks executed, as well as a .pdf version (File \textgreater{}
  Download as).
\end{itemize}

    \begin{Verbatim}[commandchars=\\\{\}]
{\color{incolor}In [{\color{incolor}1}]:} \PY{c}{\PYZsh{} Import libraries: NumPy, pandas, matplotlib}
        \PY{k+kn}{import} \PY{n+nn}{numpy} \PY{k+kn}{as} \PY{n+nn}{np}
        \PY{k+kn}{import} \PY{n+nn}{pandas} \PY{k+kn}{as} \PY{n+nn}{pd}
        \PY{k+kn}{import} \PY{n+nn}{matplotlib.pyplot} \PY{k+kn}{as} \PY{n+nn}{plt}
        
        \PY{c}{\PYZsh{} Tell iPython to include plots inline in the notebook}
        \PY{o}{\PYZpc{}}\PY{k}{matplotlib} inline
        
        \PY{c}{\PYZsh{} Read dataset}
        \PY{n}{data} \PY{o}{=} \PY{n}{pd}\PY{o}{.}\PY{n}{read\PYZus{}csv}\PY{p}{(}\PY{l+s}{\PYZdq{}}\PY{l+s}{wholesale\PYZhy{}customers.csv}\PY{l+s}{\PYZdq{}}\PY{p}{)}
        \PY{k}{print} \PY{l+s}{\PYZdq{}}\PY{l+s}{Dataset has \PYZob{}\PYZcb{} rows, \PYZob{}\PYZcb{} columns}\PY{l+s}{\PYZdq{}}\PY{o}{.}\PY{n}{format}\PY{p}{(}\PY{o}{*}\PY{n}{data}\PY{o}{.}\PY{n}{shape}\PY{p}{)}
        \PY{k}{print} \PY{n}{data}\PY{o}{.}\PY{n}{head}\PY{p}{(}\PY{p}{)}  \PY{c}{\PYZsh{} print the first 5 rows}
\end{Verbatim}

    \begin{Verbatim}[commandchars=\\\{\}]
Dataset has 440 rows, 6 columns
   Fresh  Milk  Grocery  Frozen  Detergents\_Paper  Delicatessen
0  12669  9656     7561     214              2674          1338
1   7057  9810     9568    1762              3293          1776
2   6353  8808     7684    2405              3516          7844
3  13265  1196     4221    6404               507          1788
4  22615  5410     7198    3915              1777          5185
    \end{Verbatim}

    \subsection{Feature Transformation}\label{feature-transformation}

    \textbf{1)} In this section you will be using PCA and ICA to start to
understand the structure of the data. Before doing any computations,
what do you think will show up in your computations? List one or two
ideas for what might show up as the first PCA dimensions, or what type
of vectors will show up as ICA dimensions.

    Well, ICA discovers underlying statistically independent causes. So,
while it is the case that our data consists of the yearly net price of
all products in various categories ordered by stores, there may be
independent trends consisting of combinations of the categories that get
revealed, corresponding to maybe a part of a store or type of store or
independent variance attributable to geographic position. In the context
of our lesson, we may find a component that corresponds to some shared
property of small stores that might help us solve our dilemma.

PCA will find what spread of features describes our data. So it'll tell
me what, in essence, what ratios of goods stores tend to order, and the
best first correction to that. I don't expect the PCA components
themsleves to be too useful by itself, since here the large features
will tend to swamp out the smaller ones. Fresh has the most variance,
and seems to vary greatly between customers so I expect it to show up
prominently in the first PCA component.

    \subsubsection{PCA}\label{pca}

    \begin{Verbatim}[commandchars=\\\{\}]
{\color{incolor}In [{\color{incolor}2}]:} \PY{c}{\PYZsh{} TODO: Apply PCA with the same number of dimensions as variables in the dataset}
        \PY{k+kn}{from} \PY{n+nn}{sklearn.decomposition} \PY{k+kn}{import} \PY{n}{PCA}
        \PY{n}{pca} \PY{o}{=} \PY{n}{PCA}\PY{p}{(}\PY{n}{copy}\PY{o}{=}\PY{n+nb+bp}{True}\PY{p}{)}
        \PY{n}{pca}\PY{o}{.}\PY{n}{fit}\PY{p}{(}\PY{n}{data}\PY{p}{)}
        
        \PY{c}{\PYZsh{} Print the components and the amount of variance in the data contained in each dimension}
        \PY{k}{print} \PY{n}{pca}\PY{o}{.}\PY{n}{components\PYZus{}}
        \PY{k}{print} \PY{n}{pca}\PY{o}{.}\PY{n}{explained\PYZus{}variance\PYZus{}ratio\PYZus{}}
        
        \PY{c}{\PYZsh{} Visualize the cutoff}
        \PY{n}{plt}\PY{o}{.}\PY{n}{plot}\PY{p}{(}\PY{l+m+mi}{1}\PY{o}{\PYZhy{}}\PY{n}{np}\PY{o}{.}\PY{n}{cumsum}\PY{p}{(}\PY{n}{np}\PY{o}{.}\PY{n}{insert}\PY{p}{(}\PY{n}{pca}\PY{o}{.}\PY{n}{explained\PYZus{}variance\PYZus{}ratio\PYZus{}}\PY{p}{,}\PY{l+m+mi}{0}\PY{p}{,}\PY{l+m+mi}{0}\PY{p}{)}\PY{p}{)}\PY{p}{)}
        \PY{n}{plt}\PY{o}{.}\PY{n}{axis}\PY{p}{(}\PY{p}{[}\PY{l+m+mi}{0}\PY{p}{,}\PY{l+m+mi}{6}\PY{p}{,}\PY{l+m+mi}{0}\PY{p}{,}\PY{l+m+mi}{1}\PY{p}{]}\PY{p}{)}
        \PY{n}{plt}\PY{o}{.}\PY{n}{show}\PY{p}{(}\PY{p}{)}
        
        \PY{c}{\PYZsh{} What if every feature was equally scaled?}
        \PY{n}{pca\PYZus{}prime} \PY{o}{=} \PY{n}{PCA}\PY{p}{(}\PY{p}{)}
        \PY{n}{scaled\PYZus{}data} \PY{o}{=} \PY{n}{data} \PY{o}{/} \PY{n}{data}\PY{o}{.}\PY{n}{std}\PY{p}{(}\PY{n}{axis}\PY{o}{=}\PY{l+m+mi}{0}\PY{p}{)}
        \PY{n}{pca\PYZus{}prime}\PY{o}{.}\PY{n}{fit}\PY{p}{(}\PY{n}{scaled\PYZus{}data}\PY{p}{)}
        
        \PY{k}{print} \PY{l+s}{\PYZdq{}}\PY{l+s+se}{\PYZbs{}n}\PY{l+s}{Scaled PCA}\PY{l+s+se}{\PYZbs{}n}\PY{l+s}{\PYZdq{}}
        \PY{k}{print} \PY{n}{pca\PYZus{}prime}\PY{o}{.}\PY{n}{components\PYZus{}}
        \PY{k}{print} \PY{n}{pca\PYZus{}prime}\PY{o}{.}\PY{n}{explained\PYZus{}variance\PYZus{}ratio\PYZus{}}
        
        \PY{n}{plt}\PY{o}{.}\PY{n}{plot}\PY{p}{(}\PY{l+m+mi}{1}\PY{o}{\PYZhy{}}\PY{n}{np}\PY{o}{.}\PY{n}{cumsum}\PY{p}{(}\PY{n}{np}\PY{o}{.}\PY{n}{insert}\PY{p}{(}\PY{n}{pca\PYZus{}prime}\PY{o}{.}\PY{n}{explained\PYZus{}variance\PYZus{}ratio\PYZus{}}\PY{p}{,}\PY{l+m+mi}{0}\PY{p}{,}\PY{l+m+mi}{0}\PY{p}{)}\PY{p}{)}\PY{p}{)}
        \PY{n}{plt}\PY{o}{.}\PY{n}{axis}\PY{p}{(}\PY{p}{[}\PY{l+m+mi}{0}\PY{p}{,}\PY{l+m+mi}{6}\PY{p}{,}\PY{l+m+mi}{0}\PY{p}{,}\PY{l+m+mi}{1}\PY{p}{]}\PY{p}{)}
        \PY{n}{plt}\PY{o}{.}\PY{n}{show}\PY{p}{(}\PY{p}{)}
\end{Verbatim}

    \begin{Verbatim}[commandchars=\\\{\}]
[[-0.97653685 -0.12118407 -0.06154039 -0.15236462  0.00705417 -0.06810471]
 [-0.11061386  0.51580216  0.76460638 -0.01872345  0.36535076  0.05707921]
 [-0.17855726  0.50988675 -0.27578088  0.71420037 -0.20440987  0.28321747]
 [-0.04187648 -0.64564047  0.37546049  0.64629232  0.14938013 -0.02039579]
 [ 0.015986    0.20323566 -0.1602915   0.22018612  0.20793016 -0.91707659]
 [-0.01576316  0.03349187  0.41093894 -0.01328898 -0.87128428 -0.26541687]]
[ 0.45961362  0.40517227  0.07003008  0.04402344  0.01502212  0.00613848]
    \end{Verbatim}

    \begin{center}
    \adjustimage{max size={0.9\linewidth}{0.9\paperheight}}{customer_segments_files/customer_segments_7_1.png}
    \end{center}
    { \hspace*{\fill} \\}
    
    \begin{Verbatim}[commandchars=\\\{\}]
Scaled PCA

[[-0.04288396 -0.54511832 -0.57925635 -0.05118859 -0.5486402  -0.24868198]
 [-0.52793212 -0.08316765  0.14608818 -0.61127764  0.25523316 -0.50420705]
 [-0.81225657  0.06038798 -0.10838401  0.17838615 -0.13619225  0.52390412]
 [-0.23668559 -0.08718991  0.10598745  0.76868266  0.17174406 -0.55206472]
 [ 0.04868278 -0.82657929  0.31499943  0.02793224  0.33964012  0.31470051]
 [ 0.03602539  0.03804019 -0.72174458  0.01563715  0.68589373  0.07513412]]
[ 0.44082893  0.283764    0.12334413  0.09395504  0.04761272  0.01049519]
    \end{Verbatim}

    \begin{center}
    \adjustimage{max size={0.9\linewidth}{0.9\paperheight}}{customer_segments_files/customer_segments_7_3.png}
    \end{center}
    { \hspace*{\fill} \\}
    
    \textbf{2)} How quickly does the variance drop off by dimension? If you
were to use PCA on this dataset, how many dimensions would you choose
for your analysis? Why?

    It drops off quite quickly after the first two components. The first two
components jointly describe about 85\% of the variance. The cliff after
the second principal component seems to suggest the second PC a good
cutoff point; the components afterwards aren't explaining very much
variance. On the graph you can see

    \textbf{3)} What do the dimensions seem to represent? How can you use
this information?

    The first dimension, at least, can be interpreted as the typical store
ratio of orders. What I mean by that is that if we consider all possible
feature vectors, this is the one that gets closest to describing the
data by itself, up to a constant multiplier. The first two are then just
the two feature vectors that do as above. The first component consists
almost entirely of the Fresh category, along with a small positive
correlation with the other categories as well, which seems
representative of the ordering profile of a restaurant or anywhere else
food is prepared, with its very strong focus on Fresh ingredients (Or
rather, the opposite of all this since the largest parts of the PCA
components are all negative).

The second category, however has a different interpretation. It's three
most prominent components are Grocery, Milk, and Paper/Detergents, in
that order. Those things sound exactly like what you got to a small
corner store or small grocery store to get, it has a negative
correlation for frozen and fresh because a small store like that may not
have space for these things.

Thus, despite what I initially thought would be the case, PCA does seem
to indentify small stores, their spread of orders seems to be
represented in the second PCA component; the first PCA component seems
to be an indication of what larger customers are like.

If we scale our data to unit variance before doing the PCA analysis, we
get data that looks very different on the surface, but after a bit of
thought is actually quite similar. The first PCA component (henceforth
PCA1), consists rather evenly of Milk, Grocery, and Paper, and the
second PCA component pretty evenly of Fresh, Frozen, and Deli, with a
non-trivial but small anticorrelation with Paper and Grocery. The large
differences between the magnitudes of the different main contributions
to each component nearly vanished, and it seems that the features
swapped places! What I previously described as a PCA component
corresponding to a smaller store / grocery store profile of orders is
now the first component and the restaurant style ordering profile the
second one.

We can glean about as much from this PCA analysis as the last one,
although it does reinforce the validity of the previous analysis as
being an inherent description of the most important ordering profiles
and not just an artifact of the different magnitudes of our categories.
One thing to note though, is that the stated variance described by the
second PCA analysis feels more accurate. What I mean is that it isn't
`cheating' by just describing a lot of the variance of the categories
that overwhelm the others, but it truly describes the variance in
ordering `profiles', and hence is actually more useful for categorizing
customers than the first even if the first two components describe less
of the variance (72.45\%). Notably, we could argue for the inclusion of
more components here, but having two components as a cuttoff point is
still really good, not to mention it's easier to visualize
two-dimentional data than it is to visualize data with more dimensions.

    \subsubsection{ICA}\label{ica}

    \begin{Verbatim}[commandchars=\\\{\}]
{\color{incolor}In [{\color{incolor}3}]:} \PY{c}{\PYZsh{} TODO: Fit an ICA model to the data}
        \PY{c}{\PYZsh{} Note: Adjust the data to have center at the origin first!}
        \PY{k+kn}{from} \PY{n+nn}{sklearn.decomposition} \PY{k+kn}{import} \PY{n}{FastICA}
        \PY{k+kn}{from} \PY{n+nn}{sklearn.preprocessing} \PY{k+kn}{import} \PY{n}{StandardScaler}
        
        \PY{c}{\PYZsh{} Adjusting data to have zero mean}
        \PY{n}{data\PYZus{}mean} \PY{o}{=} \PY{n}{np}\PY{o}{.}\PY{n}{mean}\PY{p}{(}\PY{n}{data}\PY{p}{)}
        \PY{n}{data\PYZus{}std} \PY{o}{=} \PY{n}{np}\PY{o}{.}\PY{n}{std}\PY{p}{(}\PY{n}{data}\PY{p}{)}
        \PY{n}{adjusted\PYZus{}data} \PY{o}{=} \PY{n}{data} \PY{o}{\PYZhy{}} \PY{n}{data\PYZus{}mean}
        
        \PY{c}{\PYZsh{} Just making sure we did this right...}
        \PY{n}{np}\PY{o}{.}\PY{n}{testing}\PY{o}{.}\PY{n}{assert\PYZus{}allclose}\PY{p}{(}\PY{n}{np}\PY{o}{.}\PY{n}{mean}\PY{p}{(}\PY{n}{adjusted\PYZus{}data}\PY{p}{)} \PY{p}{,} \PY{n}{np}\PY{o}{.}\PY{n}{zeros}\PY{p}{(}\PY{l+m+mi}{6}\PY{p}{)}\PY{p}{,} \PY{n}{atol}\PY{o}{=}\PY{l+m+mf}{1.0e\PYZhy{}10}\PY{p}{)}
        
        \PY{k}{def} \PY{n+nf}{reorder\PYZus{}rescale}\PY{p}{(}\PY{n}{mat}\PY{p}{,} \PY{n}{multiplier}\PY{o}{=}\PY{l+m+mi}{1}\PY{p}{)}\PY{p}{:}
            \PY{l+s+sd}{\PYZdq{}\PYZdq{}\PYZdq{}}
        \PY{l+s+sd}{    Reorders the rows of the matrix mat so that all of the maximal (in the absolute sense)}
        \PY{l+s+sd}{    values of each row lie on the main diagonal of the matrix, and normalizes the values of the rows}
        \PY{l+s+sd}{    such that all values on the diagonal are equal to one, for the sake of readability}
        \PY{l+s+sd}{    \PYZdq{}\PYZdq{}\PYZdq{}}
            \PY{n}{indices} \PY{o}{=} \PY{p}{[}\PY{p}{]}
            \PY{n}{retMatrix} \PY{o}{=} \PY{p}{[}\PY{p}{]}
            \PY{k}{for} \PY{n}{row} \PY{o+ow}{in} \PY{n}{mat}\PY{p}{:}
                \PY{n+nb}{max} \PY{o}{=} \PY{l+m+mf}{0.}
                \PY{n}{ele\PYZus{}num} \PY{o}{=} \PY{o}{\PYZhy{}}\PY{l+m+mi}{1}
                \PY{k}{for} \PY{n}{n}\PY{p}{,} \PY{n}{ele} \PY{o+ow}{in} \PY{n+nb}{enumerate}\PY{p}{(}\PY{n}{row}\PY{p}{)}\PY{p}{:}
                    \PY{k}{if} \PY{n+nb}{abs}\PY{p}{(}\PY{n}{ele}\PY{p}{)} \PY{o}{\PYZgt{}}\PY{o}{=} \PY{n+nb}{max}\PY{p}{:}
                        \PY{n+nb}{max} \PY{o}{=} \PY{n+nb}{abs}\PY{p}{(}\PY{n}{ele}\PY{p}{)}
                        \PY{n}{ele\PYZus{}num} \PY{o}{=} \PY{n}{n}
                \PY{n}{retMatrix}\PY{o}{.}\PY{n}{append}\PY{p}{(}\PY{n}{row}\PY{p}{)}       
                \PY{n}{indices}\PY{o}{.}\PY{n}{append}\PY{p}{(}\PY{n}{ele\PYZus{}num}\PY{p}{)}
            \PY{n}{rowOrder} \PY{o}{=} \PY{p}{[}\PY{p}{]}
            \PY{k}{for} \PY{n}{n} \PY{o+ow}{in} \PY{n+nb}{range}\PY{p}{(}\PY{n+nb}{len}\PY{p}{(}\PY{n}{indices}\PY{p}{)}\PY{p}{)}\PY{p}{:}
                \PY{k}{if} \PY{n}{n} \PY{o+ow}{in} \PY{n}{indices}\PY{p}{:}
                    \PY{n}{rowOrder}\PY{o}{.}\PY{n}{extend}\PY{p}{(}\PY{p}{[}\PY{n}{a} \PY{k}{for} \PY{n}{a}\PY{p}{,} \PY{n}{b} \PY{o+ow}{in} \PY{n+nb}{enumerate}\PY{p}{(}\PY{n}{indices}\PY{p}{)} \PY{k}{if} \PY{n}{b} \PY{o}{==} \PY{n}{n}\PY{p}{]}\PY{p}{)}
            \PY{n}{retMatrix} \PY{o}{=} \PY{p}{(}\PY{n}{np}\PY{o}{.}\PY{n}{array}\PY{p}{(}\PY{n}{retMatrix}\PY{p}{)}\PY{o}{*}\PY{n}{multiplier}\PY{p}{)}\PY{p}{[}\PY{n}{rowOrder}\PY{p}{]}
            \PY{k}{for} \PY{n}{k}\PY{p}{,} \PY{n}{row} \PY{o+ow}{in} \PY{n+nb}{enumerate}\PY{p}{(}\PY{n}{retMatrix}\PY{p}{)}\PY{p}{:}
                \PY{n}{retMatrix}\PY{p}{[}\PY{n}{k}\PY{p}{]} \PY{o}{/}\PY{o}{=} \PY{n}{retMatrix}\PY{p}{[}\PY{n}{k}\PY{p}{]}\PY{p}{[}\PY{n}{indices}\PY{p}{[}\PY{n}{rowOrder}\PY{p}{[}\PY{n}{k}\PY{p}{]}\PY{p}{]}\PY{p}{]}
            \PY{k}{return} \PY{n}{retMatrix}
        
        \PY{k}{def} \PY{n+nf}{has\PYZus{}unit\PYZus{}diagonal}\PY{p}{(}\PY{n}{matrix}\PY{p}{)}\PY{p}{:}
            \PY{n}{size} \PY{o}{=} \PY{n+nb}{len}\PY{p}{(}\PY{n}{matrix}\PY{p}{)}
            \PY{n}{tol} \PY{o}{=} \PY{l+m+mi}{10}\PY{o}{*}\PY{o}{*}\PY{p}{(}\PY{o}{\PYZhy{}}\PY{l+m+mi}{8}\PY{p}{)}
            \PY{k}{for} \PY{n}{n} \PY{o+ow}{in} \PY{n+nb}{range}\PY{p}{(}\PY{n}{size}\PY{p}{)}\PY{p}{:}
                \PY{k}{if} \PY{n+nb}{abs}\PY{p}{(}\PY{n}{matrix}\PY{p}{[}\PY{n}{n}\PY{p}{]}\PY{p}{[}\PY{n}{n}\PY{p}{]}\PY{o}{\PYZhy{}}\PY{l+m+mi}{1}\PY{p}{)} \PY{o}{\PYZgt{}} \PY{n}{tol}\PY{p}{:}
                    \PY{k}{return} \PY{n+nb+bp}{False}
            \PY{k}{return} \PY{n+nb+bp}{True}
        
        \PY{k}{def} \PY{n+nf}{run\PYZus{}ica}\PY{p}{(}\PY{n}{times}\PY{o}{=}\PY{l+m+mi}{5}\PY{p}{)}\PY{p}{:}
            \PY{l+s+sd}{\PYZdq{}\PYZdq{}\PYZdq{}Runs ICA several times, identifies common components, normalizes the rows such that their maximal element}
        \PY{l+s+sd}{    in absolute value is set to +1, and reorganizes the rows so that all elements on the diagonal are +1,}
        \PY{l+s+sd}{    before averaging\PYZdq{}\PYZdq{}\PYZdq{}}
        
            \PY{n}{means} \PY{o}{=} \PY{p}{[}\PY{p}{[}\PY{l+m+mi}{0} \PY{k}{for} \PY{n}{x} \PY{o+ow}{in} \PY{n+nb}{range}\PY{p}{(}\PY{l+m+mi}{6}\PY{p}{)}\PY{p}{]} \PY{k}{for} \PY{n}{x} \PY{o+ow}{in} \PY{n+nb}{range}\PY{p}{(}\PY{l+m+mi}{6}\PY{p}{)}\PY{p}{]}
            \PY{n}{stds} \PY{o}{=} \PY{p}{[}\PY{p}{[}\PY{l+m+mi}{0} \PY{k}{for} \PY{n}{x} \PY{o+ow}{in} \PY{n+nb}{range}\PY{p}{(}\PY{l+m+mi}{6}\PY{p}{)}\PY{p}{]} \PY{k}{for} \PY{n}{x} \PY{o+ow}{in} \PY{n+nb}{range}\PY{p}{(}\PY{l+m+mi}{6}\PY{p}{)}\PY{p}{]}
            \PY{n}{component\PYZus{}matrix\PYZus{}array} \PY{o}{=} \PY{p}{[}\PY{p}{]}
            \PY{k}{for} \PY{n}{x} \PY{o+ow}{in} \PY{n+nb}{range}\PY{p}{(}\PY{n}{times}\PY{p}{)}\PY{p}{:}
                \PY{n}{ica} \PY{o}{=} \PY{n}{FastICA}\PY{p}{(}\PY{p}{)}
                \PY{n}{source\PYZus{}data} \PY{o}{=} \PY{n}{ica}\PY{o}{.}\PY{n}{fit\PYZus{}transform}\PY{p}{(}\PY{n}{adjusted\PYZus{}data}\PY{p}{)}
                \PY{n}{source\PYZus{}data\PYZus{}std} \PY{o}{=} \PY{n}{np}\PY{o}{.}\PY{n}{std}\PY{p}{(}\PY{n}{source\PYZus{}data}\PY{p}{,} \PY{n}{axis}\PY{o}{=}\PY{l+m+mi}{0}\PY{p}{)}
                \PY{n}{component\PYZus{}matrix\PYZus{}array}\PY{o}{.}\PY{n}{append}\PY{p}{(}\PY{n}{reorder\PYZus{}rescale}\PY{p}{(}\PY{n}{ica}\PY{o}{.}\PY{n}{components\PYZus{}}\PY{p}{,} \PY{p}{[}\PY{n}{x}\PY{o}{/}\PY{n}{source\PYZus{}data\PYZus{}std}\PY{p}{[}\PY{n}{n}\PY{p}{]} \PY{k}{for} \PY{n}{n}\PY{p}{,}\PY{n}{x} \PY{o+ow}{in} \PY{n+nb}{enumerate}\PY{p}{(}\PY{n}{data\PYZus{}std}\PY{p}{)}\PY{p}{]}\PY{p}{)}\PY{p}{)}
            
            
            \PY{n}{component\PYZus{}matrix\PYZus{}array} \PY{o}{=} \PY{p}{[}\PY{n}{x} \PY{k}{for} \PY{n}{x} \PY{o+ow}{in} \PY{n}{component\PYZus{}matrix\PYZus{}array} \PY{k}{if} \PY{n}{has\PYZus{}unit\PYZus{}diagonal}\PY{p}{(}\PY{n}{x}\PY{p}{)}\PY{p}{]}
            \PY{n}{times} \PY{o}{=} \PY{n+nb}{len}\PY{p}{(}\PY{n}{component\PYZus{}matrix\PYZus{}array}\PY{p}{)}
            
            \PY{c}{\PYZsh{}Find Mean}
            \PY{k}{for} \PY{n}{matrix} \PY{o+ow}{in} \PY{n}{component\PYZus{}matrix\PYZus{}array}\PY{p}{:}
                \PY{k}{for} \PY{n}{n}\PY{p}{,} \PY{n}{row} \PY{o+ow}{in} \PY{n+nb}{enumerate}\PY{p}{(}\PY{n}{matrix}\PY{p}{)}\PY{p}{:}
                    \PY{k}{for} \PY{n}{m}\PY{p}{,} \PY{n}{element} \PY{o+ow}{in} \PY{n+nb}{enumerate}\PY{p}{(}\PY{n}{row}\PY{p}{)}\PY{p}{:}
                        \PY{n}{means}\PY{p}{[}\PY{n}{n}\PY{p}{]}\PY{p}{[}\PY{n}{m}\PY{p}{]} \PY{o}{+}\PY{o}{=} \PY{p}{(}\PY{n}{element}\PY{o}{/}\PY{n}{times}\PY{p}{)}
        
            \PY{c}{\PYZsh{}Find STD}
            \PY{k}{for} \PY{n}{matrix} \PY{o+ow}{in} \PY{n}{component\PYZus{}matrix\PYZus{}array}\PY{p}{:}
                \PY{k}{for} \PY{n}{n}\PY{p}{,} \PY{n}{row} \PY{o+ow}{in} \PY{n+nb}{enumerate}\PY{p}{(}\PY{n}{matrix}\PY{p}{)}\PY{p}{:}
                    \PY{k}{for} \PY{n}{m}\PY{p}{,} \PY{n}{element} \PY{o+ow}{in} \PY{n+nb}{enumerate}\PY{p}{(}\PY{n}{row}\PY{p}{)}\PY{p}{:}
                        \PY{n}{stds}\PY{p}{[}\PY{n}{n}\PY{p}{]}\PY{p}{[}\PY{n}{m}\PY{p}{]} \PY{o}{+}\PY{o}{=} \PY{p}{(}\PY{p}{(}\PY{n}{element} \PY{o}{\PYZhy{}} \PY{n}{means}\PY{p}{[}\PY{n}{n}\PY{p}{]}\PY{p}{[}\PY{n}{m}\PY{p}{]}\PY{p}{)}\PY{o}{*}\PY{o}{*}\PY{l+m+mi}{2} \PY{o}{/} \PY{n}{times}\PY{p}{)}
            \PY{n}{stds} \PY{o}{=} \PY{n}{np}\PY{o}{.}\PY{n}{sqrt}\PY{p}{(}\PY{n}{stds}\PY{p}{)}
            
            \PY{n}{np}\PY{o}{.}\PY{n}{set\PYZus{}printoptions}\PY{p}{(}\PY{n}{suppress}\PY{o}{=}\PY{n+nb+bp}{True}\PY{p}{)}
            
        
            \PY{n}{frac\PYZus{}std} \PY{o}{=}  \PY{p}{[}\PY{p}{[}\PY{l+m+mi}{0} \PY{k}{for} \PY{n}{x} \PY{o+ow}{in} \PY{n+nb}{range}\PY{p}{(}\PY{l+m+mi}{6}\PY{p}{)}\PY{p}{]} \PY{k}{for} \PY{n}{x} \PY{o+ow}{in} \PY{n+nb}{range}\PY{p}{(}\PY{l+m+mi}{6}\PY{p}{)}\PY{p}{]}
            \PY{k}{for} \PY{n}{n}\PY{p}{,} \PY{n}{row} \PY{o+ow}{in} \PY{n+nb}{enumerate}\PY{p}{(}\PY{n}{frac\PYZus{}std}\PY{p}{)}\PY{p}{:}
                \PY{k}{for} \PY{n}{m}\PY{p}{,} \PY{n}{element} \PY{o+ow}{in} \PY{n+nb}{enumerate}\PY{p}{(}\PY{n}{row}\PY{p}{)}\PY{p}{:}
                    \PY{n}{frac\PYZus{}std}\PY{p}{[}\PY{n}{n}\PY{p}{]}\PY{p}{[}\PY{n}{m}\PY{p}{]} \PY{o}{=} \PY{n}{np}\PY{o}{.}\PY{n}{abs}\PY{p}{(}\PY{n}{stds}\PY{p}{[}\PY{n}{n}\PY{p}{]}\PY{p}{[}\PY{n}{m}\PY{p}{]} \PY{o}{/} \PY{n}{means}\PY{p}{[}\PY{n}{n}\PY{p}{]}\PY{p}{[}\PY{n}{m}\PY{p}{]}\PY{p}{)}
            \PY{k}{print} \PY{l+s}{\PYZdq{}}\PY{l+s}{Mean Component Values}\PY{l+s}{\PYZdq{}}
            \PY{k}{print} \PY{n}{np}\PY{o}{.}\PY{n}{array}\PY{p}{(}\PY{n}{means}\PY{p}{)}
            \PY{k}{print} \PY{l+s}{\PYZdq{}}\PY{l+s}{STD of Component Values}\PY{l+s}{\PYZdq{}}
            \PY{k}{print} \PY{n}{np}\PY{o}{.}\PY{n}{array}\PY{p}{(}\PY{n}{stds}\PY{p}{)}
            \PY{k}{print} \PY{l+s}{\PYZdq{}}\PY{l+s}{Inverse Matrix}\PY{l+s}{\PYZdq{}}
            \PY{k}{print} \PY{n}{np}\PY{o}{.}\PY{n}{linalg}\PY{o}{.}\PY{n}{inv}\PY{p}{(}\PY{n}{np}\PY{o}{.}\PY{n}{array}\PY{p}{(}\PY{n}{means}\PY{p}{)}\PY{p}{)}
            \PY{n}{np}\PY{o}{.}\PY{n}{set\PYZus{}printoptions}\PY{p}{(}\PY{n}{precision}\PY{o}{=}\PY{l+m+mi}{6}\PY{p}{)}
            
            \PY{n}{components} \PY{o}{=} \PY{n}{pd}\PY{o}{.}\PY{n}{DataFrame}\PY{p}{(}\PY{n}{np}\PY{o}{.}\PY{n}{array}\PY{p}{(}\PY{n}{means}\PY{p}{)}\PY{p}{)}
            \PY{n}{components}\PY{o}{.}\PY{n}{columns} \PY{o}{=} \PY{n}{data}\PY{o}{.}\PY{n}{columns}
            \PY{n}{ax} \PY{o}{=} \PY{n}{components}\PY{o}{.}\PY{n}{plot}\PY{p}{(}\PY{n}{kind}\PY{o}{=}\PY{l+s}{\PYZdq{}}\PY{l+s}{bar}\PY{l+s}{\PYZdq{}}\PY{p}{,} \PY{n}{title}\PY{o}{=}\PY{l+s}{\PYZdq{}}\PY{l+s}{ICA components, normalized}\PY{l+s}{\PYZdq{}}\PY{p}{,} \PY{n}{figsize} \PY{o}{=} \PY{p}{(}\PY{l+m+mi}{14}\PY{p}{,}\PY{l+m+mi}{6}\PY{p}{)}\PY{p}{)}
            \PY{n}{ax}\PY{o}{.}\PY{n}{plot}\PY{p}{(}\PY{p}{)}
            \PY{k}{return} \PY{n}{components}\PY{p}{,} \PY{n}{means}
        \PY{n}{ICA\PYZus{}outputs} \PY{o}{=} \PY{n}{run\PYZus{}ica}\PY{p}{(}\PY{l+m+mi}{200}\PY{p}{)}
\end{Verbatim}

    \begin{Verbatim}[commandchars=\\\{\}]
Mean Component Values
[[ 1.         -0.12864474 -0.12950368 -0.06492207  0.20772508 -0.05629193]
 [ 0.02927879  1.         -0.7506048  -0.02342539  0.2275373  -0.23616571]
 [ 0.04005207 -0.22599926  1.          0.02761189 -0.11762155 -0.06535197]
 [-0.20201945 -0.01964551  0.13664227  1.         -0.04851627 -0.31016802]
 [ 0.02701769 -0.1326196  -0.83829251  0.05341933  1.          0.12034048]
 [-0.09599601 -0.03279289 -0.11122933 -0.04960842  0.04757085  1.        ]]
STD of Component Values
[[ 0.          0.00444574  0.01395812  0.00071753  0.01473915  0.00295652]
 [ 0.00992424  0.          0.0782913   0.00267791  0.01871597  0.0089972 ]
 [ 0.00485118  0.07857013  0.          0.00643402  0.0701574   0.02743323]
 [ 0.00047902  0.00178923  0.00408587  0.          0.00144053  0.00069358]
 [ 0.00138215  0.00994724  0.01952032  0.00085751  0.          0.00094772]
 [ 0.00031429  0.00317632  0.00386793  0.00076639  0.00186496  0.        ]]
Inverse Matrix
[[ 1.03409237  0.11215327  0.02025673  0.08911935 -0.24040291  0.1425938 ]
 [-0.0179507   1.15266089  0.74120168  0.03281004 -0.18652926  0.35227114]
 [-0.05741153  0.30568252  1.31604537 -0.0309954   0.08925961  0.13461075]
 [ 0.24526409  0.04317078 -0.06846919  1.03783627 -0.034914    0.34563244]
 [-0.10471933  0.39585287  1.19109173 -0.08691775  1.06708412  0.01005968]
 [ 0.10944299  0.06587679  0.11257566  0.06180361 -0.0717604   1.05688082]]
    \end{Verbatim}

    \begin{center}
    \adjustimage{max size={0.9\linewidth}{0.9\paperheight}}{customer_segments_files/customer_segments_13_1.png}
    \end{center}
    { \hspace*{\fill} \\}
    
    \textbf{4)} For each vector in the ICA decomposition, write a sentence
or two explaining what sort of object or property it corresponds to.
What could these components be used for?
Before proceeding towards ICA component explanations, I just want to emphasize that I am ordering the ICA components produced by their greatest feature (greatest feature in at least 80% of ICA runs for ICA1 and ICA4 since they have large features besides their main feature), and arbritrarily scaling so this greatest feature is set to +1 for the ease of analysis, since both the scale and sign of ICA components are artifacts of the method used to calculate them and not inherent properties of the data.


1) This component consists mostly of Fresh, with about 1/5 as much of Paper/Detergents and small negative numbers in the rest.  It seems to correspond with the ordering pattern of a Restaurant, with it's emphasis on fresh ingredients used in food preparation and a small amount for the paper products that are needed in conjunction with the ingredients.

2) This one is a bit difficult, but on the surface the pattern, having a large Milk component and a small Paper component would be consistent with the ordering patters of a Cafe.  The negative correlation ensures that if there's a large grocery component then we're probably dealing with a different sort of business (most likely a grocer or supermarket of some sort).

3) If we look at the inverse of the unmixing matrix, the third source component, which I'll call ICA3, figures prominently in Grocery, Milk, and Paper.  In fact, if we look at the third component of the unmixing matrix itself, we see that it contains anticorrelations with Milk, Paper/Detergents, and Deli, with a slight correlation with Frozen and Fresh.  The latter seems to be exactly what a "Grocery Store" component would be, except opposite in sign; so this component seems to be Grocery, minus the ordering profile of a grocery store, possibly since this combination increases the independence of this component compared to just Grocery by itself.

4) This component consists mostly of Frozen with moderate negative amounts of Deli, Fresh, and Grocery.  Conceivably, it could correspond to the Frozen section of a store, there maybe some deli/fresh items that actually come frozen and this distinguishes those from the kind you usually find in the frozen goods section.

5) This component is dominated by the Paper category, but interestingly enough if you look at the source data and the net money spent by each customer, it has a very weak correlation with amount spent per customer.  At the very least we can say that it doesn't distinguish between large and small customers.  That being said, the distribution does bear resemblance to what possibly a hotel would order; a good amount of paper and detergents and soaps and such to stock its rooms, with a large anticorrelation with Grocery to distinguish it from retail outlets, so I'll just say that this component describes hotel-iness.  

6) This ICA component admits a very simple explanation.  It consist almost entirely of Delicatessen with a miniscule positive correlation with Milk.  It most likely corresponds to a deli bar, either as a business in itself or as a part of a greater one.  

These components could be used for various things, not the least of which is more effectively distinguishing our customers.  While some may correspond to concepts I don't fully understand, some correspond rather clearly to concepts that I do understand, and having a high, say, ICA3 score in itself (Lots of Grocery but unlikely to be a grocery store) or a high ICA5 score ( A very hotel-like customer) is more capable of distinguishing between customers than a high Paper / Detergents ( Restaurant paper, grocery paper? ) or high Grocery score.
    \subsection{Clustering}\label{clustering}

In this section you will choose either K Means clustering or Gaussian
Mixed Models clustering, which implements expectation-maximization. Then
you will sample elements from the clusters to understand their
significance.

    \subsubsection{Choose a Cluster Type}\label{choose-a-cluster-type}

\textbf{5)} What are the advantages of using K Means clustering or
Gaussian Mixture Models?

    An advantage of the Gaussian Mixture method is that since we're assuming
our data is generated by a mixture of Gaussian distributions, every data
point has some probability that it belongs to one of the Gaussians. It
doesn't have a hard cutoff, and since in this case I don't feel like I
know very much about the data, I would rather see each point have a
component belonging to each Gaussian instead of a hard assignment to a
particular mean value. Furthermore, Gaussian mixture models more
accurately model odd cluster shapes. Gaussian Mixture models can have
pathological convergence problems, but those can usually be solved by
proper choice of parameters and preprocessing of data.

However, the Gaussian Mixture models are generally slower than k-means
to run. Furthermore, k-means is much simpler both conceptually and in
practice compared to the Gaussian Mixture methods. Also, as we can see
below, our data is not really gaussian, or at least it isn't before
undergoing some sort of transformation. If we hit our data with a
logarithm though\ldots{} It looks a lot more normally distributed! Not
to mention it spreads out the values instead of having them all be
clumped near the origin. Upon doing PCA again, we see that the results
we get are subject to the same interpretation as the ones with the
categories scaled to unit variance, so we should be fine using the
log-data in its place. The PCA components themselves have a nice
distribution after being done on log data as well.

As for which clustering method to use, after looking at histograms and
scatter plots of the log-data, it seems that K-means gives the most
meaningful cluster boundaries, since the data doesn't neatly split into
gaussians for use with a Gaussian Mixture model. Yes, the log-data is
approximately Gaussian, but having one big cluster isn't very
instructive, if we want just one center to our data we can always just
use the mean of the data, however we're looking to reveal more structure
than that, so K-means will be our clustering algorithm of choice here.

The last thing that must be decided is the number of clusters to be
used. If we look at the last of the histograms and graphs immediately
below, at the histogram for PCA1 on log data, we can immediately see how
we can split the data along either side of zero. Hence, division into
two clusters should at least be possible. There really doesn't seem to
be much of a basis for splitting into more clusters, and attempting the
actual clustering seems to back this up. This is further backed up by
the silhouette scores, which are highest for two-cluster KMeans, as can
be seen right before the answer to 7).

    \begin{Verbatim}[commandchars=\\\{\}]
{\color{incolor}In [{\color{incolor}4}]:} \PY{k}{print} \PY{l+s}{\PYZdq{}}\PY{l+s}{Histogram of categories}\PY{l+s}{\PYZdq{}}
        \PY{n}{pd}\PY{o}{.}\PY{n}{DataFrame}\PY{o}{.}\PY{n}{hist}\PY{p}{(}\PY{n}{data}\PY{p}{)}
        \PY{n}{plt}\PY{o}{.}\PY{n}{show}\PY{p}{(}\PY{p}{)}
        
        \PY{k}{print} \PY{l+s}{\PYZdq{}}\PY{l+s}{Histogram of log categories}\PY{l+s}{\PYZdq{}}
        \PY{n}{pd}\PY{o}{.}\PY{n}{DataFrame}\PY{o}{.}\PY{n}{hist}\PY{p}{(}\PY{n}{np}\PY{o}{.}\PY{n}{log}\PY{p}{(}\PY{n}{data}\PY{p}{)}\PY{p}{)}
        \PY{n}{plt}\PY{o}{.}\PY{n}{show}\PY{p}{(}\PY{p}{)}
        
        \PY{k}{print} \PY{l+s}{\PYZdq{}}\PY{l+s+se}{\PYZbs{}n}\PY{l+s}{PCA on log\PYZhy{}scaled data}\PY{l+s+se}{\PYZbs{}n}\PY{l+s}{\PYZdq{}}
        \PY{n}{pca3} \PY{o}{=} \PY{n}{PCA}\PY{p}{(}\PY{p}{)}
        \PY{n}{really\PYZus{}scaled\PYZus{}data} \PY{o}{=} \PY{n}{np}\PY{o}{.}\PY{n}{log}\PY{p}{(} \PY{n}{data} \PY{o}{/} \PY{n}{data}\PY{o}{.}\PY{n}{std}\PY{p}{(}\PY{n}{axis}\PY{o}{=}\PY{l+m+mi}{0}\PY{p}{)} \PY{p}{)}
        \PY{n}{pca3}\PY{o}{.}\PY{n}{fit}\PY{p}{(}\PY{n}{really\PYZus{}scaled\PYZus{}data}\PY{p}{)}
        
        \PY{k}{print} \PY{n}{pca3}\PY{o}{.}\PY{n}{components\PYZus{}}
        \PY{k}{print} \PY{n}{pca3}\PY{o}{.}\PY{n}{explained\PYZus{}variance\PYZus{}ratio\PYZus{}}
        
        \PY{n}{plt}\PY{o}{.}\PY{n}{plot}\PY{p}{(}\PY{l+m+mi}{1}\PY{o}{\PYZhy{}}\PY{n}{np}\PY{o}{.}\PY{n}{cumsum}\PY{p}{(}\PY{n}{np}\PY{o}{.}\PY{n}{insert}\PY{p}{(}\PY{n}{pca3}\PY{o}{.}\PY{n}{explained\PYZus{}variance\PYZus{}ratio\PYZus{}}\PY{p}{,}\PY{l+m+mi}{0}\PY{p}{,}\PY{l+m+mi}{0}\PY{p}{)}\PY{p}{)}\PY{p}{)}
        \PY{n}{plt}\PY{o}{.}\PY{n}{axis}\PY{p}{(}\PY{p}{[}\PY{l+m+mi}{0}\PY{p}{,}\PY{l+m+mi}{6}\PY{p}{,}\PY{l+m+mi}{0}\PY{p}{,}\PY{l+m+mi}{1}\PY{p}{]}\PY{p}{)}
        \PY{n}{plt}\PY{o}{.}\PY{n}{show}\PY{p}{(}\PY{p}{)}
        
        \PY{n}{pca3\PYZus{}dataframe} \PY{o}{=} \PY{n}{pd}\PY{o}{.}\PY{n}{DataFrame}\PY{p}{(}\PY{n}{data}\PY{o}{=}\PY{n}{pca3}\PY{o}{.}\PY{n}{transform}\PY{p}{(}\PY{n}{really\PYZus{}scaled\PYZus{}data}\PY{p}{)}\PY{p}{,}\PY{n}{columns}\PY{o}{=}\PY{p}{[}\PY{l+s}{\PYZsq{}}\PY{l+s}{PCA1}\PY{l+s}{\PYZsq{}}\PY{p}{,}\PY{l+s}{\PYZsq{}}\PY{l+s}{PCA2}\PY{l+s}{\PYZsq{}}\PY{p}{,}\PY{l+s}{\PYZsq{}}\PY{l+s}{PCA3}\PY{l+s}{\PYZsq{}}\PY{p}{,}\PY{l+s}{\PYZsq{}}\PY{l+s}{PCA4}\PY{l+s}{\PYZsq{}}\PY{p}{,}\PY{l+s}{\PYZsq{}}\PY{l+s}{PCA5}\PY{l+s}{\PYZsq{}}\PY{p}{,}\PY{l+s}{\PYZsq{}}\PY{l+s}{PCA6}\PY{l+s}{\PYZsq{}}\PY{p}{]}\PY{p}{)}
        \PY{n}{pd}\PY{o}{.}\PY{n}{DataFrame}\PY{o}{.}\PY{n}{hist}\PY{p}{(}\PY{n}{pca3\PYZus{}dataframe}\PY{p}{)}
        \PY{n}{plt}\PY{o}{.}\PY{n}{show}\PY{p}{(}\PY{p}{)}
\end{Verbatim}

    \begin{Verbatim}[commandchars=\\\{\}]
Histogram of categories
    \end{Verbatim}

    \begin{center}
    \adjustimage{max size={0.9\linewidth}{0.9\paperheight}}{customer_segments_files/customer_segments_19_1.png}
    \end{center}
    { \hspace*{\fill} \\}
    
    \begin{Verbatim}[commandchars=\\\{\}]
Histogram of log categories
    \end{Verbatim}

    \begin{center}
    \adjustimage{max size={0.9\linewidth}{0.9\paperheight}}{customer_segments_files/customer_segments_19_3.png}
    \end{center}
    { \hspace*{\fill} \\}
    
    \begin{Verbatim}[commandchars=\\\{\}]
PCA on log-scaled data

[[-0.173717  0.394463  0.454364 -0.172196  0.745515  0.149436]
 [ 0.685136  0.162399  0.069379  0.487691  0.041912  0.509709]
 [ 0.673512 -0.033068  0.032183 -0.284811  0.218244 -0.644597]
 [-0.214326  0.018558  0.064487  0.806083  0.19005  -0.51349 ]
 [ 0.000822 -0.722288 -0.347993  0.036243  0.563665  0.195366]
 [ 0.029276 -0.543035  0.813908  0.017448 -0.202444  0.022832]]
[ 0.442374  0.276571  0.116173  0.096177  0.045757  0.022947]
    \end{Verbatim}

    \begin{center}
    \adjustimage{max size={0.9\linewidth}{0.9\paperheight}}{customer_segments_files/customer_segments_19_5.png}
    \end{center}
    { \hspace*{\fill} \\}
    
    \begin{center}
    \adjustimage{max size={0.9\linewidth}{0.9\paperheight}}{customer_segments_files/customer_segments_19_6.png}
    \end{center}
    { \hspace*{\fill} \\}
    
    \textbf{6)} Below is some starter code to help you visualize some
cluster data. The visualization is based on
\href{http://scikit-learn.org/stable/auto_examples/cluster/plot_kmeans_digits.html}{this
demo} from the sklearn documentation.

    \begin{Verbatim}[commandchars=\\\{\}]
{\color{incolor}In [{\color{incolor}5}]:} \PY{c}{\PYZsh{} Import clustering modules}
        \PY{k+kn}{from} \PY{n+nn}{sklearn.cluster} \PY{k+kn}{import} \PY{n}{KMeans}
        \PY{k+kn}{from} \PY{n+nn}{sklearn.mixture} \PY{k+kn}{import} \PY{n}{GMM}
\end{Verbatim}

    \begin{Verbatim}[commandchars=\\\{\}]
{\color{incolor}In [{\color{incolor}6}]:} \PY{c}{\PYZsh{} TODO: First we reduce the data to two dimensions using PCA to capture variation}
        \PY{n}{pca2} \PY{o}{=} \PY{n}{PCA}\PY{p}{(}\PY{n}{n\PYZus{}components}\PY{o}{=}\PY{l+m+mi}{2}\PY{p}{)}
        \PY{n}{reduced\PYZus{}data} \PY{o}{=} \PY{n}{pca2}\PY{o}{.}\PY{n}{fit\PYZus{}transform}\PY{p}{(}\PY{n}{np}\PY{o}{.}\PY{n}{log}\PY{p}{(}\PY{n}{data} \PY{o}{/} \PY{n}{data}\PY{o}{.}\PY{n}{std}\PY{p}{(}\PY{n}{axis}\PY{o}{=}\PY{l+m+mi}{0}\PY{p}{)}\PY{p}{)}\PY{p}{)}
        \PY{k}{print} \PY{n}{reduced\PYZus{}data}\PY{p}{[}\PY{p}{:}\PY{l+m+mi}{10}\PY{p}{]}  \PY{c}{\PYZsh{} print upto 10 elements}
\end{Verbatim}

    \begin{Verbatim}[commandchars=\\\{\}]
[[ 1.750985  0.070515]
 [ 1.800365  0.869754]
 [ 1.893736  1.676621]
 [-1.127313  1.457982]
 [ 0.796711  2.460866]
 [ 1.083824  0.392942]
 [ 1.13261  -0.201602]
 [ 1.582474  0.968196]
 [ 0.871825 -0.596759]
 [ 2.888883  0.741103]]
    \end{Verbatim}

    \begin{Verbatim}[commandchars=\\\{\}]
{\color{incolor}In [{\color{incolor}7}]:} \PY{c}{\PYZsh{} Condensated several steps into one convenient function to graph the results of a clustering.}
        \PY{k+kn}{from} \PY{n+nn}{sklearn} \PY{k+kn}{import} \PY{n}{metrics}
        
        \PY{k}{def} \PY{n+nf}{graph\PYZus{}clusters}\PY{p}{(}\PY{n}{reduced\PYZus{}data}\PY{p}{,} \PY{n}{clusters}\PY{p}{)}\PY{p}{:}     
            \PY{c}{\PYZsh{} Plot the decision boundary by building a mesh grid to populate a graph.}
            \PY{n}{x\PYZus{}min}\PY{p}{,} \PY{n}{x\PYZus{}max} \PY{o}{=} \PY{n}{reduced\PYZus{}data}\PY{p}{[}\PY{p}{:}\PY{p}{,} \PY{l+m+mi}{0}\PY{p}{]}\PY{o}{.}\PY{n}{min}\PY{p}{(}\PY{p}{)} \PY{o}{\PYZhy{}} \PY{l+m+mi}{1}\PY{p}{,} \PY{n}{reduced\PYZus{}data}\PY{p}{[}\PY{p}{:}\PY{p}{,} \PY{l+m+mi}{0}\PY{p}{]}\PY{o}{.}\PY{n}{max}\PY{p}{(}\PY{p}{)} \PY{o}{+} \PY{l+m+mi}{1}
            \PY{n}{y\PYZus{}min}\PY{p}{,} \PY{n}{y\PYZus{}max} \PY{o}{=} \PY{n}{reduced\PYZus{}data}\PY{p}{[}\PY{p}{:}\PY{p}{,} \PY{l+m+mi}{1}\PY{p}{]}\PY{o}{.}\PY{n}{min}\PY{p}{(}\PY{p}{)} \PY{o}{\PYZhy{}} \PY{l+m+mi}{1}\PY{p}{,} \PY{n}{reduced\PYZus{}data}\PY{p}{[}\PY{p}{:}\PY{p}{,} \PY{l+m+mi}{1}\PY{p}{]}\PY{o}{.}\PY{n}{max}\PY{p}{(}\PY{p}{)} \PY{o}{+} \PY{l+m+mi}{1}
            \PY{n}{hx} \PY{o}{=} \PY{p}{(}\PY{n}{x\PYZus{}max}\PY{o}{\PYZhy{}}\PY{n}{x\PYZus{}min}\PY{p}{)}\PY{o}{/}\PY{l+m+mf}{1000.}
            \PY{n}{hy} \PY{o}{=} \PY{p}{(}\PY{n}{y\PYZus{}max}\PY{o}{\PYZhy{}}\PY{n}{y\PYZus{}min}\PY{p}{)}\PY{o}{/}\PY{l+m+mf}{1000.}
            \PY{n}{xx}\PY{p}{,} \PY{n}{yy} \PY{o}{=} \PY{n}{np}\PY{o}{.}\PY{n}{meshgrid}\PY{p}{(}\PY{n}{np}\PY{o}{.}\PY{n}{arange}\PY{p}{(}\PY{n}{x\PYZus{}min}\PY{p}{,} \PY{n}{x\PYZus{}max}\PY{p}{,} \PY{n}{hx}\PY{p}{)}\PY{p}{,} \PY{n}{np}\PY{o}{.}\PY{n}{arange}\PY{p}{(}\PY{n}{y\PYZus{}min}\PY{p}{,} \PY{n}{y\PYZus{}max}\PY{p}{,} \PY{n}{hy}\PY{p}{)}\PY{p}{)}
        
            \PY{c}{\PYZsh{} Obtain labels for each point in mesh. Use last trained model.}
            \PY{n}{Z} \PY{o}{=} \PY{n}{clusters}\PY{o}{.}\PY{n}{predict}\PY{p}{(}\PY{n}{np}\PY{o}{.}\PY{n}{c\PYZus{}}\PY{p}{[}\PY{n}{xx}\PY{o}{.}\PY{n}{ravel}\PY{p}{(}\PY{p}{)}\PY{p}{,} \PY{n}{yy}\PY{o}{.}\PY{n}{ravel}\PY{p}{(}\PY{p}{)}\PY{p}{]}\PY{p}{)}
            
            \PY{c}{\PYZsh{} TODO: Find the centroids for KMeans or the cluster means for GMM }
            \PY{n}{centroids} \PY{o}{=} \PY{n}{clusters}\PY{o}{.}\PY{n}{means\PYZus{}} \PY{k}{if} \PY{n+nb}{isinstance}\PY{p}{(}\PY{n}{clusters}\PY{p}{,}\PY{n}{GMM}\PY{p}{)} \PY{k}{else} \PY{n}{clusters}\PY{o}{.}\PY{n}{cluster\PYZus{}centers\PYZus{}}
            \PY{k}{print} \PY{l+s}{\PYZdq{}}\PY{l+s}{\PYZhy{}\PYZhy{} Centers \PYZhy{}\PYZhy{}}\PY{l+s}{\PYZdq{}}
            \PY{k}{print} \PY{n}{centroids}
            \PY{k}{print} \PY{l+s}{\PYZdq{}}\PY{l+s}{\PYZhy{}\PYZhy{} Centers in Original Coordinates \PYZhy{}\PYZhy{}}\PY{l+s}{\PYZdq{}}
            \PY{n}{vectors} \PY{o}{=} \PY{n}{pca2}\PY{o}{.}\PY{n}{inverse\PYZus{}transform}\PY{p}{(}\PY{n}{centroids}\PY{p}{)}
            \PY{k}{for} \PY{n}{v} \PY{o+ow}{in} \PY{n}{vectors}\PY{p}{:}
                \PY{k}{print} \PY{p}{(}\PY{n}{np}\PY{o}{.}\PY{n}{exp}\PY{p}{(}\PY{n}{v}\PY{p}{)} \PY{o}{*} \PY{n}{data}\PY{o}{.}\PY{n}{std}\PY{p}{(}\PY{n}{axis}\PY{o}{=}\PY{l+m+mi}{0}\PY{p}{)}\PY{p}{)}
            \PY{k}{print} \PY{l+s}{\PYZdq{}}\PY{l+s}{ Silhouette Coefficient: \PYZob{}\PYZcb{}}\PY{l+s}{\PYZdq{}}\PY{o}{.}\PY{n}{format}\PY{p}{(}\PY{n}{metrics}\PY{o}{.}\PY{n}{silhouette\PYZus{}score}\PY{p}{(}\PY{n}{reduced\PYZus{}data}\PY{p}{,} \PY{n}{clusters}\PY{o}{.}\PY{n}{labels\PYZus{}}\PY{p}{)}\PY{p}{)}
            \PY{c}{\PYZsh{} Put the result into a color plot}
            \PY{n}{Z} \PY{o}{=} \PY{n}{Z}\PY{o}{.}\PY{n}{reshape}\PY{p}{(}\PY{n}{xx}\PY{o}{.}\PY{n}{shape}\PY{p}{)}
            \PY{n}{plt}\PY{o}{.}\PY{n}{figure}\PY{p}{(}\PY{l+m+mi}{1}\PY{p}{)}
            \PY{n}{plt}\PY{o}{.}\PY{n}{clf}\PY{p}{(}\PY{p}{)}
            \PY{n}{plt}\PY{o}{.}\PY{n}{imshow}\PY{p}{(}\PY{n}{Z}\PY{p}{,} \PY{n}{interpolation}\PY{o}{=}\PY{l+s}{\PYZsq{}}\PY{l+s}{nearest}\PY{l+s}{\PYZsq{}}\PY{p}{,}
                       \PY{n}{extent}\PY{o}{=}\PY{p}{(}\PY{n}{xx}\PY{o}{.}\PY{n}{min}\PY{p}{(}\PY{p}{)}\PY{p}{,} \PY{n}{xx}\PY{o}{.}\PY{n}{max}\PY{p}{(}\PY{p}{)}\PY{p}{,} \PY{n}{yy}\PY{o}{.}\PY{n}{min}\PY{p}{(}\PY{p}{)}\PY{p}{,} \PY{n}{yy}\PY{o}{.}\PY{n}{max}\PY{p}{(}\PY{p}{)}\PY{p}{)}\PY{p}{,}
                       \PY{n}{cmap}\PY{o}{=}\PY{n}{plt}\PY{o}{.}\PY{n}{cm}\PY{o}{.}\PY{n}{Paired}\PY{p}{,}
                       \PY{n}{aspect}\PY{o}{=}\PY{l+s}{\PYZsq{}}\PY{l+s}{auto}\PY{l+s}{\PYZsq{}}\PY{p}{,} \PY{n}{origin}\PY{o}{=}\PY{l+s}{\PYZsq{}}\PY{l+s}{lower}\PY{l+s}{\PYZsq{}}\PY{p}{)}
        
        
            \PY{n}{plt}\PY{o}{.}\PY{n}{plot}\PY{p}{(}\PY{n}{reduced\PYZus{}data}\PY{p}{[}\PY{p}{:}\PY{p}{,} \PY{l+m+mi}{0}\PY{p}{]}\PY{p}{,} \PY{n}{reduced\PYZus{}data}\PY{p}{[}\PY{p}{:}\PY{p}{,} \PY{l+m+mi}{1}\PY{p}{]}\PY{p}{,} \PY{l+s}{\PYZsq{}}\PY{l+s}{k.}\PY{l+s}{\PYZsq{}}\PY{p}{,} \PY{n}{markersize}\PY{o}{=}\PY{l+m+mi}{2}\PY{p}{)}
            \PY{n}{plt}\PY{o}{.}\PY{n}{scatter}\PY{p}{(}\PY{n}{centroids}\PY{p}{[}\PY{p}{:}\PY{p}{,} \PY{l+m+mi}{0}\PY{p}{]}\PY{p}{,} \PY{n}{centroids}\PY{p}{[}\PY{p}{:}\PY{p}{,} \PY{l+m+mi}{1}\PY{p}{]}\PY{p}{,}
                        \PY{n}{marker}\PY{o}{=}\PY{l+s}{\PYZsq{}}\PY{l+s}{x}\PY{l+s}{\PYZsq{}}\PY{p}{,} \PY{n}{s}\PY{o}{=}\PY{l+m+mi}{169}\PY{p}{,} \PY{n}{linewidths}\PY{o}{=}\PY{l+m+mi}{3}\PY{p}{,}
                        \PY{n}{color}\PY{o}{=}\PY{l+s}{\PYZsq{}}\PY{l+s}{w}\PY{l+s}{\PYZsq{}}\PY{p}{,} \PY{n}{zorder}\PY{o}{=}\PY{l+m+mi}{10}\PY{p}{)}
            \PY{n}{plt}\PY{o}{.}\PY{n}{title}\PY{p}{(}\PY{l+s}{\PYZsq{}}\PY{l+s}{Clustering on the wholesale grocery dataset (PCA\PYZhy{}reduced data)}\PY{l+s+se}{\PYZbs{}n}\PY{l+s}{\PYZsq{}}
                      \PY{l+s}{\PYZsq{}}\PY{l+s}{Centroids are marked with white cross}\PY{l+s}{\PYZsq{}}\PY{p}{)}
            \PY{n}{plt}\PY{o}{.}\PY{n}{xlim}\PY{p}{(}\PY{n}{x\PYZus{}min}\PY{p}{,} \PY{n}{x\PYZus{}max}\PY{p}{)}
            \PY{n}{plt}\PY{o}{.}\PY{n}{ylim}\PY{p}{(}\PY{n}{y\PYZus{}min}\PY{p}{,} \PY{n}{y\PYZus{}max}\PY{p}{)}
            \PY{c}{\PYZsh{}plt.xticks(())}
            \PY{c}{\PYZsh{}plt.yticks(())}
            \PY{n}{plt}\PY{o}{.}\PY{n}{axis}\PY{p}{(}\PY{p}{[}\PY{n}{x\PYZus{}min}\PY{p}{,} \PY{n}{x\PYZus{}max}\PY{p}{,} \PY{n}{y\PYZus{}min}\PY{p}{,} \PY{n}{y\PYZus{}max}\PY{p}{]}\PY{p}{)}
            \PY{n}{plt}\PY{o}{.}\PY{n}{show}\PY{p}{(}\PY{p}{)}
\end{Verbatim}

    \begin{Verbatim}[commandchars=\\\{\}]
{\color{incolor}In [{\color{incolor}8}]:} \PY{c}{\PYZsh{} Try three different clusterings}
        
        \PY{n}{meanClusters2} \PY{o}{=} \PY{n}{KMeans}\PY{p}{(}\PY{n}{n\PYZus{}clusters}\PY{o}{=}\PY{l+m+mi}{2}\PY{p}{)}
        \PY{n}{meanClusters2}\PY{o}{.}\PY{n}{fit}\PY{p}{(}\PY{n}{reduced\PYZus{}data}\PY{p}{)}
        \PY{n}{graph\PYZus{}clusters}\PY{p}{(}\PY{n}{reduced\PYZus{}data}\PY{p}{,} \PY{n}{meanClusters2}\PY{p}{)}
        
        \PY{n}{meanClusters3} \PY{o}{=} \PY{n}{KMeans}\PY{p}{(}\PY{n}{n\PYZus{}clusters}\PY{o}{=}\PY{l+m+mi}{3}\PY{p}{)}
        \PY{n}{meanClusters3}\PY{o}{.}\PY{n}{fit}\PY{p}{(}\PY{n}{reduced\PYZus{}data}\PY{p}{)}
        \PY{n}{graph\PYZus{}clusters}\PY{p}{(}\PY{n}{reduced\PYZus{}data}\PY{p}{,} \PY{n}{meanClusters3}\PY{p}{)}
        
        \PY{n}{meanClusters4} \PY{o}{=} \PY{n}{KMeans}\PY{p}{(}\PY{n}{n\PYZus{}clusters}\PY{o}{=}\PY{l+m+mi}{4}\PY{p}{)}
        \PY{n}{meanClusters4}\PY{o}{.}\PY{n}{fit}\PY{p}{(}\PY{n}{reduced\PYZus{}data}\PY{p}{)}
        \PY{n}{graph\PYZus{}clusters}\PY{p}{(}\PY{n}{reduced\PYZus{}data}\PY{p}{,} \PY{n}{meanClusters4}\PY{p}{)}
\end{Verbatim}

    \begin{Verbatim}[commandchars=\\\{\}]
-- Centers --
[[-1.504351  0.164155]
 [ 2.21427  -0.241622]]
-- Centers in Original Coordinates --
Fresh               8994.280204
Milk                1908.923867
Grocery             2366.187501
Frozen              2080.775809
Detergents\_Paper     290.422230
Delicatessen         681.298712
dtype: float64
Fresh                3570.082284
Milk                 7748.505394
Grocery             12462.617477
Frozen                899.888171
Detergents\_Paper     4567.020517
Delicatessen          965.713272
dtype: float64
 Silhouette Coefficient: 0.419166083203
    \end{Verbatim}

    \begin{center}
    \adjustimage{max size={0.9\linewidth}{0.9\paperheight}}{customer_segments_files/customer_segments_24_1.png}
    \end{center}
    { \hspace*{\fill} \\}
    
    \begin{Verbatim}[commandchars=\\\{\}]
-- Centers --
[[ 2.003276 -2.200361]
 [ 1.597532  1.279829]
 [-1.775741 -0.026178]]
-- Centers in Original Coordinates --
Fresh                967.770457
Milk                5187.090060
Grocery             9884.521631
Frozen               359.007506
Detergents\_Paper    3594.726669
Delicatessen         344.795136
dtype: float64
Fresh               11269.803880
Milk                 7778.022894
Grocery             10465.321936
Frozen               2101.629494
Detergents\_Paper     3073.571144
Delicatessen         1912.566267
dtype: float64
Fresh               8275.750671
Milk                1662.922124
Grocery             2064.246238
Frozen              1987.046075
Detergents\_Paper     235.340502
Delicatessen         593.733697
dtype: float64
 Silhouette Coefficient: 0.392358689958
    \end{Verbatim}

    \begin{center}
    \adjustimage{max size={0.9\linewidth}{0.9\paperheight}}{customer_segments_files/customer_segments_24_3.png}
    \end{center}
    { \hspace*{\fill} \\}
    
    \begin{Verbatim}[commandchars=\\\{\}]
-- Centers --
[[-0.923558  1.066035]
 [ 2.292966 -2.401511]
 [-2.316957 -1.041468]
 [ 2.158875  0.810268]]
-- Centers in Original Coordinates --
Fresh               15083.551412
Milk                 2779.040094
Grocery              3279.670102
Frozen               2922.878492
Detergents\_Paper      465.040735
Delicatessen         1176.721822
dtype: float64
Fresh                 801.797118
Milk                 5628.136707
Grocery             11118.821543
Frozen                309.624609
Detergents\_Paper     4423.831030
Delicatessen          324.963392
dtype: float64
Fresh               4534.600386
Milk                1139.061267
Grocery             1504.433597
Frozen              1329.358943
Detergents\_Paper     150.655604
Delicatessen         326.376641
dtype: float64
Fresh                7410.509020
Milk                 8993.255212
Grocery             13072.924579
Frozen               1517.480395
Detergents\_Paper     4579.774068
Delicatessen         1637.206215
dtype: float64
 Silhouette Coefficient: 0.332881619258
    \end{Verbatim}

    \begin{center}
    \adjustimage{max size={0.9\linewidth}{0.9\paperheight}}{customer_segments_files/customer_segments_24_5.png}
    \end{center}
    { \hspace*{\fill} \\}
    
    \textbf{7)} What are the central objects in each cluster? Describe them
as customers.

    If we look at one center, it has abnormally high amounts in the Grocery,
Milk, and Detergents/Paper categories. They are either near or exceed
the actual maximal values in the data. The reason for this is that the
PCA was done on the log unit-variance scaled data, and the variacne
explained doesn't cut off as quickly as it does on the unmodified data,
hence it's understandable that the first two PCA components produce
clusters with centers that can't actually be customers. However, it's
still very interpretable, and in this case it can be interpreted as the
typical Grocery store, since those three categories would be expected to
dominate in such a case.

For the other center, it consists of a large amount of Fresh along with
a good amount of Frozen. This center is to the left and covers a greater
range of data points than the other. However, despite that I still think
that we can interpret this cluster center as corresponding to a typical
customer in Food Service, ordering mostly Fresh and Frozen ingredients
in order to cook and serve.

    \subsubsection{Conclusions}\label{conclusions}

** 8)** Which of these techniques did you feel gave you the most insight
into the data?

    I feel that, in conjunction with more domain knowledge and more
knowledge of the data and where it comes from, these techniques would
have been much more useful. But as is, I'm confronted mostly with
numbers and six inprecise labels. Out of all the techniques used here, I
would say clustering was the least useful since it revealed very little
and since the data didn't divide that neatly into clusters.

The interpretations given for the ICA component vectors are a bit
tenuous, but it seems like it was the most promising out of all of them.
With greater access to information regarding the origin of the data, it
may have been possible to determine with greater accuracy what the ICA
components actually correspond to. However, as is, the ICA analysis did
prove somewhat useful. From our analysis, it seems that grocery
stores/supermarkets, hotels, restaurants, delicatessens are among the
company's customers. Although the identities of the customers behind the
ICA components is part conjecture, the conjectures involved are at least
plausible, and with greater future data that might identify customers,
or with possible investigation of groups of customers, even more light
can be shed on what the ICA components actually correspond to.

    \textbf{9)} How would you use that technique to help the company design
new experiments?

    Whenever we perform A/B testing, it is vital that we evenly distribute
our testing across vital customer attributes. In the context of this
assignment, this can mean that we may want to distribute testing across
small and large customers, and possibly among the different types of
customers.

The most basic thing we can do is to segment our A/B testing to contain
representatives from all three of the clusters found above, so we know
how changes affect all of them and don't let the fact that, say cluster
2 contains most of our customers, obscure the effects of a test on
members of cluster 1 and cluster 3.

However, as for ICA, the technique that I said gave the most insight
into the data in the previous question, we could use it as follows: We
could form (overlapping) bins by categorizing whether a customer has an
above or below average value in a certain ICA component, then insist
that any sample of customers we choose draw from all of those bins and
either ensure that an A/B test succeeds in all of the bins, or to at the
very least pay attention to what happens in every bin into the
conclusions of the test.

For any customer, calculate for it a vector with values taken from
\{0,1\} depending on whether it has greater than or less than the mean
value in the ith ICA component. Every possible vector combination should
exist in the data, for the ICA components, by construction, should not
correlate with each other, and since the number of possibile vector
values (2\^{}6 = 64) is much less than the total number of data points.
We can then just make sure to pick customers from each one of these
sixty-four bins when we do our testing, and if we see certain bins have
different results from others, we can use them to see where these
differing results occur. Maybe most of the bins with low ICA3 get
different results than the rest so possibly grocery stores are
especially impacted by a change, or a combination of factors could be
responsible as well like in the case where low ICA2 coupled with high
ICA4 seems to cause a change being tested by A/B testing to not perform
well.

    \textbf{10)} How would you use that data to help you predict future
customer needs?

    The usefulness of the data for predicting future customer needs is
varied. If used in conjunction with surverys, we can use the knowledge
we've found so far in order to properly segment customers into
categories which we can them sample from and hence find targeted advice
to the different groups to which they belong.

If we want to use supervised learning techniques, this sort of
exploratory analysis that we conducted above is extremely useful. ICA
provides more effective components to learn on, and in the case that we
have too many dimensions to effectively run our algorithms, we can use
PCA to reduce the amount of dimensions to maybe three or four before
feeding it into, say, a neural network or a support vector machine.
While some methods, like the boosting methods I employed in the last
project, don't need data preprocessing, others, like support vector
machines, benefit greatly from their use. However, the dimensionality
reduction we can do with PCA would be great for fitting a regression,
which also has the benefit of having easily interpreted coefficients
(unlike the ICA -\textgreater{} neural network or SVM combination).


    % Add a bibliography block to the postdoc
    
    
    
    \end{document}
